\section{Qubit\-Vertex Struct Reference}
\label{structQubitVertex}\index{QubitVertex@{QubitVertex}}
{\tt \#include $<$graphsim.h$>$}

\subsection*{Public Member Functions}
\begin{CompactItemize}
\item 
{\bf Qubit\-Vertex} (void)
\end{CompactItemize}
\subsection*{Public Attributes}
\begin{CompactItemize}
\item 
{\bf Loc\-Cliff\-Op} {\bf byprod}
\item 
hash\_\-set$<$ {\bf Vertex\-Index} $>$ {\bf neighbors}
\end{CompactItemize}


\subsection{Detailed Description}
A {\bf Graph\-Register}{\rm (p.\,\pageref{classGraphRegister})} object maintains a list of its vertices (qubits), each described by an object of the class Qubit\-Vertex described here. 



\subsection{Constructor \& Destructor Documentation}
\index{QubitVertex@{Qubit\-Vertex}!QubitVertex@{QubitVertex}}
\index{QubitVertex@{QubitVertex}!QubitVertex@{Qubit\-Vertex}}
\subsubsection{\setlength{\rightskip}{0pt plus 5cm}Qubit\-Vertex::Qubit\-Vertex (void)\hspace{0.3cm}{\tt  [inline]}}\label{structQubitVertex_a0}


Upon construction, a qubit vertex is initialised with the Hadamard operation as VOp, and with wmpty neighbor list. This makes it represent a $|$0$>$. 

\subsection{Member Data Documentation}
\index{QubitVertex@{Qubit\-Vertex}!byprod@{byprod}}
\index{byprod@{byprod}!QubitVertex@{Qubit\-Vertex}}
\subsubsection{\setlength{\rightskip}{0pt plus 5cm}{\bf Loc\-Cliff\-Op} {\bf Qubit\-Vertex::byprod}}\label{structQubitVertex_o0}


byprod is the vertex operator (VOp) associated with the qubit (the name stems from the term 'byproduct operator' used for the similar concept in the one-way quantum computer. \index{QubitVertex@{Qubit\-Vertex}!neighbors@{neighbors}}
\index{neighbors@{neighbors}!QubitVertex@{Qubit\-Vertex}}
\subsubsection{\setlength{\rightskip}{0pt plus 5cm}hash\_\-set$<${\bf Vertex\-Index}$>$ {\bf Qubit\-Vertex::neighbors}}\label{structQubitVertex_o1}


neigbors is the adjacency list for this vertex 

The documentation for this struct was generated from the following file:\begin{CompactItemize}
\item 
{\bf graphsim.h}\end{CompactItemize}
